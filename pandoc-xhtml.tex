% Pandoc-XHTML is a environment for ConTeXt to transform HTML5 output from
% pandoc into ConTeXt (to produce the final PDF).
% Copyright (C) 2016 Pablo Rodríguez
%                    Andreas Schneider
%
% This work is free software; you can redistribute it and/or
% modify it under the terms of the GNU General Public License
% as published by the Free Software Foundation; either version 2
% of the License, or (at your option) any later version.
%
% This work is distributed in the hope that it will be useful,
% but WITHOUT ANY WARRANTY; without even the implied warranty of
% MERCHANTABILITY or FITNESS FOR A PARTICULAR PURPOSE.  See the
% GNU General Public License for more details.
%
% You should have received a copy of the GNU General Public License
% along with this work; if not, write to the Free Software
% Foundation, Inc., 51 Franklin Street, Fifth Floor, Boston, MA  02110-1301, USA.

\startluacode
    usercode = usercode or {}
    usercode.authors = usercode.authors or {}

    local gsub = string.gsub

    function xml.functions.collectauthors(root)
        print("collect authors called.")
        for e in xml.collected(root, "/h2[contains(@class,'author')]") do
            table.insert(usercode.authors, xml.text(e))
            print("Inserting: "..xml.text(e))
        end
        print("Test: "..utilities.parsers.array_to_string(usercode.authors, ", "))
    end

    function xml.expressions.idstring(str)
        return type(str) == "string" and gsub(str,"^#","") or ""
    end

    function xml.expressions.stringlength(str)
        return string.len(str)
    end
\stopluacode

\startxmlsetups xml:pandoc
    \xmlsetsetup{\xmldocument}
        {html|body|div|h1|h2|h3|h4|h5|h6|p|blockquote|span|em|q|b|strong|a|ul|ol|li|dl|dt|dd|hr|br|sup|sub|code|img|EQ}
        {xml:*}

    \xmlsetsetup{\xmldocument}
        {pre/code}
        {xml:pre:code}

    \xmlsetsetup{\xmldocument}
        {head/style}
        {}

    \xmlsetsetup{\xmldocument}
        {head/meta[@name='date']}
        {}

    \xmlsetsetup{\xmldocument}
        {head/title}
        {xml:doc:title}

    \xmlsetsetup{\xmldocument}
        {head/meta[@name='subtitle']}
        {xml:doc:subtitle}

    \xmlsetsetup{\xmldocument}
        {head/meta[@name='author']}
        {xml:doc:author}

    \xmlsetsetup{\xmldocument}
        {h2[contains(@class,'author')]}
        {xml:title:author}

    \xmlsetsetup{\xmldocument}
        {(div|span|code)[@lang]}
        {xml:lang}

    \xmlsetsetup{\xmldocument}
        {[contains(@class,'smcp')]}
        {xml:smallcaps}

    \xmlsetsetup{\xmldocument}
        {[contains(@class,'part')]/h1}
        {xml:part}

    \xmlsetsetup{\xmldocument}
        {[contains(@class,'part')]/[contains(@class,'level2')]/h2}
        {xml:h1}

    \xmlsetsetup{\xmldocument}
        {[contains(@class,'part')]/[contains(@class,'level2')]/[contains(@class,'level3')]/h3}
        {xml:h2}

    \xmlsetsetup{\xmldocument}
        {[contains(@class,'part')]/[contains(@class,'level2')]/[contains(@class,'level3')]/[contains(@class,'level4')]/h4}
        {xml:h3}

    \xmlsetsetup{\xmldocument}
        {[contains(@class,'part')]/[contains(@class,'level2')]/[contains(@class,'level3')]/[contains(@class,'level4')]/[contains(@class,'level5')]/h5}
        {xml:h4}

    \xmlsetsetup{\xmldocument}
        {[contains(@class,'hidden')]/h1}
        {xml:hidden:h1}

    \xmlsetsetup{\xmldocument}
        {[@id='header']}
        {xml:titlepage}

    \xmlsetsetup{\xmldocument}
        {[@id='copyright' or contains(@class,'rights') or contains(@class,'copyright')]}
        {xml:rights}

    \xmlsetsetup{\xmldocument}
        {[@id='dedication' or contains(@class,'dedication')]}
        {xml:dedication}

    \xmlsetsetup{\xmldocument}
        {[@id='epigraph' or contains(@class,'epigraph')]}
        {xml:epigraph}

    \xmlsetsetup{\xmldocument}
        {[@id='frontmatter' or contains(@class,'frontmatter')]}
        {xml:frontmatter}

    \xmlsetsetup{\xmldocument}
        {[@id='bodymatter' or contains(@class,'bodymatter')]}
        {xml:bodymatter}

    \xmlsetsetup{\xmldocument}
        {[@id='backmatter' or contains(@class,'backmatter')]}
        {xml:backmatter}

    \xmlsetsetup{\xmldocument}
        {[@id='appendices' or contains(@class,'appendices')]}
        {xml:appendices}

    \xmlsetsetup{\xmldocument}
        {[@id='colophon' or contains(@class,'colophon')]}
        {xml:colophon}

    % Table of Contents
    \xmlsetsetup{\xmldocument}
        {[@id='toc' or contains(@class,'toc')]}
        {xml:toc}

    % Table of Figures
    \xmlsetsetup{\xmldocument}
        {[@id='tof' or contains(@class,'tof')]}
        {xml:tof}

    % Table of Tables
    \xmlsetsetup{\xmldocument}
        {[@id='tot' or contains(@class,'tot')]}
        {xml:tot}

    \xmlsetsetup{\xmldocument}
        {[@class='signature-date']}
        {xml:signaturedate}

    \xmlsetsetup{\xmldocument}
        {[@class='signature-author']}
        {xml:signatureauthor}

    \xmlsetsetup{\xmldocument}
        {span[@label or contains(@class,'tex\letterpercent-logo')]}
        {xml:tex:logo}

    \xmlsetsetup{\xmldocument}
        {a[contains(@href,'\letterhat\letterhash[A-Za-z]')]}
        {xml:internal:link}

    \xmlsetsetup{\xmldocument}
        {a[@class='footnoteRef']}
        {xml:footnote:ref}

    \xmlsetsetup{\xmldocument}
        {div[@class='footnotes']}
        {}

    \xmlsetsetup{\xmldocument}
        {a[@class='uri']}
        {xml:autolink}

    \xmlsetsetup{\xmldocument}
        {a[@class='footnoteBack']}
        {}

    \xmlsetsetup{\xmldocument}
        {a[text()='↩']}
        {}

    \xmlsetsetup{\xmldocument}
        {[contains(@class,'tex\letterpercent-logo')]/(sup|sub)}
        {xml:texlogo:sup:sub}

    \xmlsetsetup{\xmldocument}
        {ol[@style='list-style-type: decimal']}
        {xml:ol}

    \xmlsetsetup{\xmldocument}
        {ol[@style='list-style-type: lower-alpha']}
        {xml:loweralphalist}

    \xmlsetsetup{\xmldocument}
        {ol[@style='list-style-type: upper-alpha']}
        {xml:upperalphalist}

    \xmlsetsetup{\xmldocument}
        {ol[@style='list-style-type: lower-roman']}
        {xml:lowerromanlist}

    \xmlsetsetup{\xmldocument}
        {ol[@style='list-style-type: upper-roman']}
        {xml:upperromanlist}

    \xmlsetsetup{\xmldocument}
        {ol[@style='list-style-type: lower-greek']}
        {xml:lowergreeklist}

    \xmlsetsetup{\xmldocument}
        {div[contains(@class,'options')]/dl/dt}
        {xml:dt_options}

    \xmlsetsetup{\xmldocument}
        {div[contains(@class,'options')]/dl/dl}
        {xml:dl_options}

    \xmlsetsetup{\xmldocument}
        {div[contains(@class,'none')]/dl/dt}
        {xml:dt_none}

    \xmlsetsetup{\xmldocument}
        {div[contains(@class,'none')]/dl/dl}
        {xml:dl_none}

    \xmlsetsetup{\xmldocument}
        {juh:footnotes}
        {xml:juh:footnotes}

    % Table handling
    \xmlsetsetup{\xmldocument}
        {table|thead|tbody|tr}
        {xml:table:*}

    \xmlsetsetup{\xmldocument}
        {th|td}
        {xml:table:column}

    \xmlsetsetup{\xmldocument}
        {table/caption}
        {}

    \xmlsetsetup{\xmldocument}
        {div[contains(@class,'figure')]}
        {xml:img:float}

    \xmlsetsetup{\xmldocument}
        {div[contains(@class,'figure')]/p[@class='caption']}
        {}

    \xmlsetsetup{\xmldocument}
        {span[@class='math inline']}
        {xml:math:inline}

    \xmlsetsetup{\xmldocument}
        {EQ[@ENV='math']}
        {xml:math:eq}

    \xmlsetsetup{\xmldocument}
        {div[contains(@class,'placefloats')]}
        {xml:placefloats}

    % Skip empty h1-headings. This is handy when you want to use h1-headings for special
    % markups. (Like starting the frontmatter.)
    \xmlsetsetup{\xmldocument}
        {h1[stringlength(text()) == 0]}
        {-}

    \xmlsetsetup{\xmldocument}
        {span[contains(@class,'symbol')]}
        {xml:tex:symbol}

\stopxmlsetups

\xmlregistersetup{xml:pandoc}

\startxmlsetups xml:html
    \mainlanguage[\xmlatt{#1}{lang}]
    \xmlflush{#1}
\stopxmlsetups

\startxmlsetups xml:body
    \xmlflush{#1}
\stopxmlsetups

\startxmlsetups xml:doc:title
    \setupinteraction[title={\xmlflush{#1}}]
\stopxmlsetups

\startxmlsetups xml:doc:subtitle
    \setupinteraction[subtitle={\xmlatt{#1}{content}}]
\stopxmlsetups

\startxmlsetups xml:title:author
    \xmlflush{#1}
\stopxmlsetups

\startxmlsetups xml:titlepage
    % Set variables so we can reuse them in the document.
    \setevalue{doc:title}{\xmltext{#1}{h1[@class='title']}}
    \setevalue{doc:subtitle}{\xmltext{#1}{h1[@class='subtitle']}}
    \setevalue{doc:publisher}{\xmltext{#1}{p[@class='publisher']}}

    \xmlfunction{#1}{collectauthors}

    % Layout the cover page
    \startfirstmatter
    \setupinteraction
        [author={\docauthors{, }},
         keyword={\xmldoifelse{#1}{p[@class='publisher']/a}
                    {\xmltext{#1}{p[@class='publisher']/a}}
                    {\xmltext{#1}{p[@class='publisher']}}}]
    \startmode[coverpage]
        \makecoverpage
    \stopmode

    % Layout the title page
    \starttitlepagemakeup
        \def\TitlePage{%
        \startmode[*en,*uk]\hiddentitle{[Title page]}\stopmode%
        \startmode[*es]\hiddentitle{[Título]}\stopmode%
        \startmode[*deo,*de,*de-de,*de-at,*de-ch]\hiddentitle{[Titelseite]}\stopmode%
        }
        \doifdefined{hiddentitle}{\TitlePage}
        \vfill

        {\itd\getvalue{doc:title}\par}\blank[big]
        {\resetbreakpoints\itb\setupinterlinespace \getvalue{doc:subtitle}\par}\blank[3*big]
        {\sc \docauthors{\crlf{}}}

        \vfill
        \xmltext{#1}{h3[@class='date']}
        \vfill
        \xmltext{#1}{p[@class='publisher']}
    \stoptitlepagemakeup
\stopxmlsetups

\startxmlsetups xml:rights
    \startcopyrightmakeup
    \xmlflush{#1}
    \stopcopyrightmakeup
\stopxmlsetups

\startxmlsetups xml:dedication
    \startdedicationmakeup
    \xmlflush{#1}
    \stopdedicationmakeup
\stopxmlsetups

\startxmlsetups xml:epigraph
    \startepigraphmakeup
    \xmlflush{#1}
    \stopepigraphmakeup
\stopxmlsetups

\startxmlsetups xml:frontmatter
    \doifmode{*whatcomesfirst}{\stopfirstmatter}

    \startfrontmatter
    \xmlflush{#1}
\stopxmlsetups

\startxmlsetups xml:bodymatter
    \stopfrontmatter

    \startbodymatter
    \xmlflush{#1}
\stopxmlsetups

\startxmlsetups xml:backmatter
    \doifmode{*bodypart}{\stopbodymatter}
    \startbackmatter
    \xmlflush{#1}
\stopxmlsetups

\startxmlsetups xml:appendices
    \doifmodeelse{*bodypart}
        {\stopbodymatter}
        {\doifmode{*backpart}{\stopbackmatter}}
    \startappendices
    \xmlflush{#1}
\stopxmlsetups

\startxmlsetups xml:juh:footnotes
    \def\JUHFootnotesTitle{%
    \startmode[*en,*uk]\chapter{Notes}\stopmode%
    \startmode[*es]\chapter{Notas}\stopmode%
    \startmode[*deo,*de,*de-de,*de-at,*de-ch]\chapter{Anmerkungen}\stopmode%
    \startmode[*fr]\chapter{Notes}\stopmode}
    \JUHFootnotesTitle\placefootnotes
\stopxmlsetups

\startxmlsetups xml:colophon
    \doifmode{*bodypart}{\stopbodymatter}
    \doifmode{*appendix}{\stopappendices}
    \doifmode{*backpart}{\stopbackmatter}
    \startlastmatter
    \startnotmode[nofootnotes,juh:footnotes]\startmode[endnotes]
    \def\FootnotesTitle{%
    \startmode[*en,*uk]\chapter{Notes}\stopmode%
    \startmode[*es]\chapter{Notas}\stopmode%
    \startmode[*deo,*de,*de-de,*de-at,*de-ch]\chapter{Anmerkungen}\stopmode%
    \startmode[*fr]\chapter{Notes}\stopmode}
    \FootnotesTitle\placefootnotes
    \stopmode\stopnotmode

    \startcolophonmakeup
    \xmlflush{#1}
    \stopcolophonmakeup
    \stoplastmatter
\stopxmlsetups

\startxmlsetups xml:div
    \xmlflush{#1}
\stopxmlsetups

\startxmlsetups xml:lang
    \bgroup
        \language[\xmlatt{#1}{lang}]\xmlsetup{#1}{xml:\xmltag{#1}}
    \egroup
\stopxmlsetups

\startxmlsetups xml:part
    \part[\xmlattribute{#1}{..//div}{id}]{\xmlflush{#1}}
\stopxmlsetups

% Dynamically build the heading. This essentially takes the given level (>=2),
% prepends the appropriate amount of "sub" and then decides, whether it should
% be a section (numbered) or subject (unnumbered).
% Therefore \createheading[4]{...} will lead to \subsubsection or \subsubsubject.
\def\createheading[#1]#2{
    % Setup variables
    \edef\prefix{}
    \newcount\X
    \X=#1

    % Subtract 2, since is is our lowest possible level which should not have
    % any "sub" prefixed.
    \advance \X by -2

    % Iterate the given amount and prepend "sub".
    \dorecurse{\X}{\edef\prefix{sub\prefix}}

    % Decide if we want numbering or not.
    \xmldoifelse{#2}{..[contains(@class,'unnumbered')]}{%
        \csname\prefix subject\endcsname[\xmlatt{#2}{id}]{\xmlflush{#2}}%
    }{%
        \csname\prefix section\endcsname[\xmlatt{#2}{id}]{\xmlflush{#2}}%
    }
}

\startxmlsetups xml:h1
    %~ \startchapter[bookmark={\xmltext{\xmldocument}{../h1}}, title={\xmlflush{#1}}]\stopchapter
    \xmldoifelse{#1}{.[contains(@class,'unnumbered')]}{%
        \title[\xmlatt{#1}{id}]{\xmlflush{#1}}%
    }{%
        \chapter[\xmlatt{#1}{id}]{\xmlflush{#1}}%
    }
\stopxmlsetups

\startxmlsetups xml:hidden:h1
    \doifdefined{hiddentitle}
    {\hiddentitle[\xmlattribute{#1}{..//div}{id}]{\xmlflush{#1}}}
\stopxmlsetups

\startxmlsetups xml:h2
    \createheading[2]{#1}
\stopxmlsetups

\startxmlsetups xml:h3
    \createheading[3]{#1}
\stopxmlsetups

\startxmlsetups xml:h4
    \createheading[4]{#1}
\stopxmlsetups

\startxmlsetups xml:h5
    \createheading[5]{#1}
\stopxmlsetups

\startxmlsetups xml:h6
    \createheading[6]{#1}
\stopxmlsetups

\startxmlsetups xml:p
    \xmldoifnotselfempty {#1} {
        \dontleavehmode
        \ignorespaces
        \xmlflush{#1}
        \removeunwantedspaces
    }
    \par
\stopxmlsetups

\startxmlsetups xml:blockquote
    %~ check both modes for mainlanguage and language \doifmodeelse{*en}{\blank\startnarrow\setupindenting[yes,next]\xmlflush{#1}\stopnarrow\blank}    {\blank\startnarrow\xmlflush{#1}\stopnarrow\blank}
    \startblockquote\xmlflush{#1}\stopblockquote
\stopxmlsetups

\startxmlsetups xml:signaturedate
    \blank[samepage, big]
    \startalign[flushright]
        \xmlflush{#1}
    \stopalign
\stopxmlsetups

\startxmlsetups xml:signatureauthor
    \blank[samepage, big]
    \startalign[flushright]
        \bgroup\sc\xmlflush{#1}\egroup
    \stopalign
\stopxmlsetups

\startxmlsetups xml:span
   \xmlflush{#1}
\stopxmlsetups

\startxmlsetups xml:em
    \bgroup\em\xmlflush{#1}\egroup
\stopxmlsetups

\startxmlsetups xml:q
    \quotation{\xmlflush{#1}}
\stopxmlsetups

\startxmlsetups xml:b
    \bgroup\bf\xmlflush{#1}\egroup
\stopxmlsetups

\startxmlsetups xml:smallcaps
    \bgroup\sc\xmlflush{#1}\egroup
\stopxmlsetups

\startxmlsetups xml:pre:code
    \bgroup\setupinterlinespace[line=2.8ex]\xmlprettyprint{#1}{none}\egroup
\stopxmlsetups

\startxmlsetups xml:code % need extra code to work
    \bgroup\sethyphenationfeatures[underscore]\tt\xmlflushspacewise{#1}\egroup
\stopxmlsetups

\startxmlsetups xml:strong % used for sans-serif instead of bold
    \doifmodeelse{strong:ssandnotbf}
    {\bgroup\ss\xmlflush{#1}\egroup}
    {\bgroup\bf\xmlflush{#1}\egroup}
\stopxmlsetups

\startxmlsetups xml:a
    \href{\xmlatt{#1}{href}}{\xmlflush{#1}}
\stopxmlsetups

\startxmlsetups xml:internal:link
    \xmldoifnotselfempty{#1}{\xmlflush{#1} }
    \in[\xmlrefatt{#1}{href}]%
\stopxmlsetups

\startxmlsetups xml:autolink
    \mypersonalurl{\xmlflush{#1}}
\stopxmlsetups

\startxmlsetups xml:footnote:set
     \startfootnote
         \xmlflush{#1}
     \stopfootnote
\stopxmlsetups

\startnotmode[nofootnotes]
\startxmlsetups xml:footnote:ref
    \xmlfilter{main}{div[@class='footnotes']/ol/li[@id=idstring('\xmlatt{#1}{href}')]/command(xml:footnote:set)}
\stopxmlsetups
\stopnotmode


\startxmlsetups xml:ol
    \xmldoifelse{#1}{/li/p}{
        \doifnumberelse{\xmlatt{#1}{start}}{
            \startitemize[n][start=\xmlatt{#1}{start}]
                \xmlflush{#1}
            \stopitemize
            } {
            \startitemize[n]
                \xmlflush{#1}
            \stopitemize}
        } {
        \doifnumberelse{\xmlatt{#1}{start}}{
            \startitemize[n,packed][start=\xmlatt{#1}{start}]
                \xmlflush{#1}
            \stopitemize
            } {
            \startitemize[n,packed]
                \xmlflush{#1}
            \stopitemize}
    }
\stopxmlsetups

\startxmlsetups xml:loweralphalist
    \xmldoifelse{#1}{/li/p}{
        \doifnumberelse{\xmlatt{#1}{start}}{
            \startitemize[a][start=\xmlatt{#1}{start}]
                \xmlflush{#1}
            \stopitemize
            } {
            \startitemize[a]
                \xmlflush{#1}
            \stopitemize}
        } {
        \doifnumberelse{\xmlatt{#1}{start}}{
            \startitemize[a,packed][start=\xmlatt{#1}{start}]
                \xmlflush{#1}
            \stopitemize
            } {
            \startitemize[a,packed]
                \xmlflush{#1}
            \stopitemize}
    }
\stopxmlsetups

\startxmlsetups xml:upperalphalist
    \xmldoifelse{#1}{/li/p}{
        \doifnumberelse{\xmlatt{#1}{start}}{
            \startitemize[A][start=\xmlatt{#1}{start}]
                \xmlflush{#1}
            \stopitemize
            } {
            \startitemize[A]
                \xmlflush{#1}
            \stopitemize}
        } {
        \doifnumberelse{\xmlatt{#1}{start}}{
            \startitemize[A,packed][start=\xmlatt{#1}{start}]
                \xmlflush{#1}
            \stopitemize
            } {
            \startitemize[A,packed]
                \xmlflush{#1}
            \stopitemize}
    }
\stopxmlsetups

\startxmlsetups xml:lowerromanlist
    \xmldoifelse{#1}{/li/p}{
        \doifnumberelse{\xmlatt{#1}{start}}{
            \startitemize[r][start=\xmlatt{#1}{start}]
                \xmlflush{#1}
            \stopitemize
            } {
            \startitemize[r]
                \xmlflush{#1}
            \stopitemize}
        } {
        \doifnumberelse{\xmlatt{#1}{start}}{
            \startitemize[r,packed][start=\xmlatt{#1}{start}]
                \xmlflush{#1}
            \stopitemize
            } {
            \startitemize[r,packed]
                \xmlflush{#1}
            \stopitemize}
    }
\stopxmlsetups

\startxmlsetups xml:upperromanlist
    \xmldoifelse{#1}{/li/p}{
        \doifnumberelse{\xmlatt{#1}{start}}{
            \startitemize[R][start=\xmlatt{#1}{start}]
                \xmlflush{#1}
            \stopitemize
            } {
            \startitemize[R]
                \xmlflush{#1}
            \stopitemize}
        } {
        \doifnumberelse{\xmlatt{#1}{start}}{
            \startitemize[R,packed][start=\xmlatt{#1}{start}]
                \xmlflush{#1}
            \stopitemize
            } {
            \startitemize[R,packed]
                \xmlflush{#1}
            \stopitemize}
    }
\stopxmlsetups

\startxmlsetups xml:lowergreeklist % not implemented by pandoc
    \xmldoifelse{#1}{/li/p}{
        \doifnumberelse{\xmlatt{#1}{start}}{
            \startitemize[g][start=\xmlatt{#1}{start}]
                \xmlflush{#1}
            \stopitemize
            } {
            \startitemize[g]
                \xmlflush{#1}
            \stopitemize}
        } {
        \doifnumberelse{\xmlatt{#1}{start}}{
            \startitemize[g,packed][start=\xmlatt{#1}{start}]
                \xmlflush{#1}
            \stopitemize
            } {
            \startitemize[g,packed]
                \xmlflush{#1}
            \stopitemize}
    }
\stopxmlsetups

\startxmlsetups xml:ul
     \xmldoifelse{#1}{/li/p}{
         \startitemize
             \xmlflush{#1}
         \stopitemize
     } {
         \startitemize[packed]
             \xmlflush{#1}
         \stopitemize
     }
\stopxmlsetups


\startxmlsetups xml:li
    \startitem
        \xmlflush{#1}
    \stopitem
\stopxmlsetups

\startxmlsetups xml:dl
    \xmlflush{#1}
\stopxmlsetups

\startxmlsetups xml:dt
    \startdescription{\xmlflush{#1}}
\stopxmlsetups

\startxmlsetups xml:dd
    \xmlflush{#1}
    \stopdescription
\stopxmlsetups

\startxmlsetups xml:dt_options
    \startoptions{\xmlflush{#1}}
\stopxmlsetups

\startxmlsetups xml:dd_options
    \xmlflush{#1}
    \stopoptions
\stopxmlsetups

\startxmlsetups xml:dd_none
    \xmlflush{#1}
    \stopnone
\stopxmlsetups

\startxmlsetups xml:hr
    \blank\hrule\blank
\stopxmlsetups

\startxmlsetups xml:br
     \crlf
\stopxmlsetups

\startxmlsetups xml:sup
     \high{\xmlflush{#1}}
\stopxmlsetups

\startxmlsetups xml:texlogo:sup:sub
     \xmlflush{#1}
\stopxmlsetups

\startxmlsetups xml:sub
     \low{\xmlflush{#1}}
\stopxmlsetups

%%% this element is the way to get the TeX and ConTeXt logos
%%% markdown tagging:
%%% <span label="tex">TeX</span>
%%% Sp you get standard text in other formats and logos with ConTeXt

\startxmlsetups xml:logo
    \executeifdefined{\xmlatt{#1}{label}}{{\tttf ???}}
\stopxmlsetups

%%% alternate TeX logos
%%% <span class="tex-logo">TeX</span>

\startxmlsetups xml:tex:logo
    \executeifdefined{\xmlflush{#1}}{{\tttf ???}}
\stopxmlsetups

%%%%%%%%%% Additions

\startxmlsetups xml:table:table
    \xmldoif{#1}{/caption/float}{%
        \placetable%
            [\xmlattribute{#1}{/caption/float}{options}]%
            [\xmlattribute{#1}{/caption/float}{id}]%
            {\xmltext{#1}{/caption}}%
            \bgroup}
    
    \bTABLE
    \xmlflush{#1}
    \eTABLE

    \xmldoif{#1}{/caption/float}{\egroup}
\stopxmlsetups

\startxmlsetups xml:table:thead
    \bTABLEhead
    \xmlflush{#1}
    \eTABLEhead
\stopxmlsetups

\startxmlsetups xml:table:tbody
    \bTABLEbody
    \xmlflush{#1}
    \eTABLEbody
\stopxmlsetups

\startxmlsetups xml:table:tr
    \bTR \xmlflush{#1} \eTR
\stopxmlsetups

\startxmlsetups xml:table:column
    \bTD \xmlflush{#1} \eTD
\stopxmlsetups

\startxmlsetups xml:toc
    \completecontent
\stopxmlsetups

\startxmlsetups xml:tot
    \completelistoftables
\stopxmlsetups

\startxmlsetups xml:tof
    \completelistoffigures
\stopxmlsetups

\startluacode
    function usercode.processfigure(file, t)
        -- Parse the options from our input first.
        local args = utilities.parsers.settings_to_hash(t)
        -- We don't allow the width or height to exceed the text dimensions.
        -- "sp" needs to be added, since dimensions are stored that way.
        args["maxwidth"] = tex.dimen["textwidth"].."sp"
        args["maxheight"] = tex.dimen["textheight"].."sp"

        -- Finally: call externalfigure for our file with out arguments.
        context.externalfigure({file}, args)
    end
\stopluacode

\startxmlsetups xml:img
    \ctxlua{usercode.processfigure("\xmlatt{#1}{src}", "\xmlatt{#1}{options}")}
\stopxmlsetups

\startxmlsetups xml:img:float
    \placefigure[][\xmlattribute{#1}{/img}{id}]{\xmltext{#1}{/p[@class='caption']}}{
        \xmlflush{#1}
    }
\stopxmlsetups

\startxmlsetups xml:math:inline
    \xmlflushcontext{#1}
\stopxmlsetups

\startxmlsetups xml:math:eq
    \math{\xmlflushcontext{#1}}
\stopxmlsetups

\startxmlsetups xml:placefloats
    \placefloats
\stopxmlsetups

\startxmlsetups xml:tex:symbol
    \symbol[\xmlatt{#1}{set}][\xmlflush{#1}]
\stopxmlsetups

%%%%%%%%%% Necessary setups for pandoc-xhtml to work
\definesectionblock
    [whatcomesfirst]
    [firstmatter]

\setupsectionblock
    [firstmatter]
    [page=no, after=\page]

\definesectionblock
    [tocpart]
    [tableofcontents]
    [number=no]

\setupsectionblock
    [tableofcontents]
    [page=no,
     after=\page]

\startsectionblockenvironment[tocpart]
    \setuppagenumbering[location=]
    %~ \setupalign[middle]
    \setupinterlinespace[line=2.8ex]
\stopsectionblockenvironment

\def\docauthors#1{%
    \cldcontext{utilities.parsers.array_to_string(usercode.authors,"\luaescapestring{#1}")}
}

\def\makecoverpage{
    \startcoverpagemakeup
        \def\CoverPageTitle{%
            \startmode[*en,*uk]\hiddentitle{[Cover page]}\stopmode%
            \startmode[*es]\hiddentitle{[Portada]}\stopmode%
            \startmode[*deo,*de,*de-de,*de-at,*de-ch]\hiddentitle{[Bildtitelseite]}\stopmode%
        }
        \doifdefined{hiddentitle}{\CoverPageTitle}
        \setupbodyfont[svb]
        {\setupbodyfont[25pt]\sc \docauthors{\crlf{}} }
        \vfill
        \scale[width=\textwidth]{\bf \getvalue{doc:title}}

        \blank[small]\startalign [flushright]
        \dontleavehmode\scale[width=.75\textwidth]{\it \getvalue{doc:subtitle}}
        \stopalign
        \vfill
        \startalign[center]
        \dontleavehmode\CoverImage
        \stopalign
        \vfill
        \startalign[flushright]
        \dontleavehmode\scale[width=.75\textwidth]{\getvalue{doc:publisher}}
        \stopalign
    \stopcoverpagemakeup
}

%%% This is a way of having a cover image
%%% But you might want to have a new cover.
%%% Default image from (edited by me):
%%% https://openclipart.org/detail/216016/
%%% TypographyTribute can be found at:
%%% http://www.dafont.com/typographytribute.font

\def\CoverImageFile{}%\externalfigure[TypeWriter][width=\textwidth]}
\definefontfamily[ornamenta][rm][TypographyTribute]
\def\CoverImageFont{\scale[width=\textwidth]{\ornamenta F}}
\def\CoverImage{\doifsomethingelse{\CoverImageFile}{\CoverImageFile}{\CoverImageFont}}

%%% This a simple method to have interactive hyperlinks

\def\mypersonalurl#1{\bgroup\tt\hw\goto{\hyphenatedurl{#1}}[url(#1)]\egroup}
\unexpanded\def\href#1#2{\goto{#2} [url(#1)]}

%%% Two samples for definition lists

\definedescription[description][hang=fit, width=fit, headstyle=em, align=justify, margin=yes]
\definedescription[options][hang=fit, width=fit, headstyle=tt, align=justify, location=left, margin=yes]
\definedescription[none][width=fit, headstyle=, align=justify, location=left, margin=yes]

%%% These start/stop structures are created (defined) to avoid
%%% crashes with the current setup. They need to be configured
%%% (setup) when if we want to use them (otherwise we get only
%%% defaults).

\definestartstop[blockquote]

%%% These makeups are created (defined) to avoid crashes with
%%% the current setup. They need to be configured (setup)
%%% when if we want to use them (otherwise we get only defaults).

\definemakeup[coverpage]
\definemakeup[titlepage]
\definemakeup[copyright]
\definemakeup[dedication]
\definemakeup[epigraph]
\definemakeup[colophon]

%%% This command removes unwanted code from bookmarks.

\appendtoks
    \let\em\space
    \let\tt\space
    \let\ss\space
    \let\bgroup\space
    \let\egroup\space
    \let\sethyphenationfeatures\space
    \let\language\space
    \let\en\space
    \let\agr\space
    \let\es\space
    \let\de\space
    \let\fr\space
    \let\it\space
    \let\la\space
    \let\deo\space
\to \simplifiedcommands

%%% the current setup. They need to be configured (setup)
%%% when if we want to use them (otherwise we get only defaults).

%%% These are language synonyms, so that XML values can match
%%% ConTeXt values (btw, ConTeXt seems to have less languages
%%% than LaTeX)

\installlanguage [af-ZA] [af]
\installlanguage [ar-DZ] [ar-dz]
\installlanguage [ar-IQ] [ar-iq]
\installlanguage [ar-JO] [ar-jo]
\installlanguage [ar-LB] [ar-lb]
\installlanguage [ar-MA] [ar-ma]
\installlanguage [ar-SY] [ar-sy]
\installlanguage [bg-BG] [bg]
\installlanguage [ca-ES] [ca]
\installlanguage [cy-UK] [cy]
\installlanguage [cz-CZ] [cz]
\installlanguage [da-DK] [da]
\installlanguage [de-1901] [deo]
\installlanguage [de-AT] [de-at]
\installlanguage [de-CH] [de-ch]
\installlanguage [de-DE] [de-de]
\installlanguage [el] [gr]
\installlanguage [en-UK] [uk]
\installlanguage [en-GB] [uk]
\installlanguage [en-US] [en]
\installlanguage [es-ES] [es]
\installlanguage [et-EE] [et]
\installlanguage [eu-ES] [eu]
\installlanguage [fi-FI] [fi]
\installlanguage [fr-CA] [fr]
\installlanguage [fr-FR] [fr]
\installlanguage [grc] [agr]
\installlanguage [he] [il]
\installlanguage [he-IL] [il]
\installlanguage [hr-HR] [hr]
\installlanguage [hu-HU] [hu]
\installlanguage [is-IS] [is]
\installlanguage [it-IT] [it]
\installlanguage [jp] [ja]
\installlanguage [jp-JP] [ja]
\installlanguage [nb-NO] [nb]
\installlanguage [nl-NL] [nl]
\installlanguage [nn-NO] [nn]
\installlanguage [no-NO] [no]
\installlanguage [pl-PL] [pl]
\installlanguage [pt-BR] [pt]
\installlanguage [pt-PT] [pt]
\installlanguage [ro-RO] [ro]
\installlanguage [ru-RU] [ru]
\installlanguage [sk-SK] [sk]
\installlanguage [sl-SL] [sl]
\installlanguage [sv-SE] [sv]
\installlanguage [tr-TR] [tr]
\installlanguage [uk] [ua]
\installlanguage [uk-UA] [ua]
\installlanguage [vi] [vn]
\installlanguage [vi-VN] [vn]